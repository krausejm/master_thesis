\documentclass{scdpg}
\begin{document}
\scBookLanguage{de}
\begin{scAbstract}
%\scNoUseTeX
\scLanguage{en}
\scTitle{Recent polarization observable results in $\eta$- and $\eta'$-photoproduction off the proton.}
\scAuthor{*}{Jakob}{Krause}{1}
\scCollaborationName{CBELSA/TAPS}
\scAffiliation{1}{Helmholtz-Institut für Strahlen- und Kernphysik, Universität Bonn.}
\scBeginText
While generally good agreement exists for low lying hadronic resonances, especially for high masses there are much more resonances predicted than actually found. This is also known as the problem of the "missing resonances", indicating the poor understanding of QCD in the non-perturbative region. Studying meson photoproduction off the nucleon promises to give further insight into the nucleon excitation spectrum. The analysis thereof requires partial wave analysis (PWA) to identify contributing resonances. It is essential to measure single and double polarization observables in order to find unambiguous PWA solutions. The CBELSA/TAPS experiment located at the electron stretcher accelerator ELSA in Bonn is dedicated to measuring different polarization observables in meson photoproduction employing a polarized photon beam and a polarized target. 

This talk will present preliminary results concerning the polarization observable $\Sigma$ in the reactions $\gamma p \to p \eta$ and $\gamma p \to p \eta'$ measured at the CBELSA/TAPS experiment. ??This work is supported by ...??
\scEndText
\scConference{Mainz 2022}
\scPart{HK}
\scContributionType{Gruppenbericht;Group Report}
\scTopic{Hadronenstruktur und -spektroskopie}
\scEmail{krause@hiskp.uni-bonn.de}
\scCountry{Germany}

\end{scAbstract}
\end{document}
