% !TEX root = master_thesis.tex
\chapter{Summary}

In order to eliminate ambiguities in the invariant amplitudes of pseudoscalar meson photoproduction polarization observables need to be measured, ultimately adding to the understanding of the strong interaction in the non-perturbative regime. Utilizing a linearly polarized photon beam and an unpolarized target, data were taken at the CBELSA/TAPS experiment located at the accelerator ELSA. This allowed the extraction of the beam  asymmetry $\Sigma$ in the beam energy range $\SI{1100}{\mega\eV}<E_\gamma<\SI{1800}{\mega\eV}$.

The use of \textsc{Bayesian} methods for the determination of polarization observables was tested by reproducing high precision data for the beam asymmetry $\Sigma$ in $\eta$ photoproduction that were obtained at the CBELSA/TAPS experiment using a binned $\chi^2$ fit and an unbinned maximum likelihood fit. Instead of point estimates with statistical errors, a \textsc{Bayesian} fit result contains marginal posterior distributions for each fit parameter, significantly increasing the amount of information obtained from the fit. In order to acquire these distributions it is necessary to rely on sophisticated \textsc{Markov}-chain Monte Carlo methods that increase the computational cost and also require detailed diagnostics. It was found that the \textsc{Bayesian} approach gives equivalent results as the traditional frequentist approach and no systematic error is introduced by the choice of fitting method. Since polarization observables are used as input of PWA calculations, providing distributions is a distinct advantage because the user may choose errors derived from the distributions in any way that is appropriate for the chosen model. Also, the \textsc{Bayesian} approach allows to truncate the posterior distributions to the physically allowed parameter space in a convenient way that is not applicable to traditional $\chi^2$ or maximum-likelihood fits. Since the application of \textsc{Bayesian} methods to determine polarization observables they were used to determine the beam asymmetry for the $p\eta'$ final state in the following.
 
Using the neutral decay $\eta'\to\gamma\gamma$ the $p\eta'$ final state was selected with significant background contributions from the $p2\pi^0\to p4\gamma$ and $p\pi^0\eta\to p4\gamma$ final states as has been found investigating Monte Carlo simulations; during reconstruction two photons get lost either because they have not enough energy to surpass the reconstruction threshold for particle energy depositions or because they are combined to one cluster with a nearby high energy photon. Although the acceptance for both reactions after the complete event selection is $A\ll1$, the large difference in the production cross sections as well as the branching ratio to an only-photon final state gave total background contributions of up to $45\%$. No sensible background reducing cuts have been found. In total $8\cdot10^3$ $p\eta'$ events were retained from which the beam asymmetry $\Sigma$ was determined using an unbinned maximum likelihood fit and an unbinned \textsc{Bayesian} fit. Both methods yielded equivalent results. Hereby the beam asymmetry for $2\pi^0$ production was known in the same kinematic binning which allowed to correct the estimated values and distributions according to the respective amount of background contamination. As opposed to the maximum likelihood fit, the \textsc{Bayesian} fit allowed to include the background contributions as an additional fit parameter inherently to the chosen probabilistic model. The corrected values and distributions are in fair agreement with previous measurements taken at CLAS using the charged decay $\eta'\to\pi^+\pi^-\eta$ and also agree with PWA predictions within their statistical error. In the energy region $\SI{1100}{\mega\eV}<E_\gamma<\SI{1800}{\mega\eV}$ the new results for the beam asymmetry in $\eta'$ photoproduction off the proton obtained from CBELSA/TAPS data provide the only addition to existing results from CLAS.  However, the precision of the CBELSA/TAPS data can not compete with the CLAS data due to the relatively small number of events that could be reconstructed in the chosen decay channel. Nevertheless, agreement with previous measurements increases the informative value of both datasets that thus contribute to the identification of high-mass $N^*$ resonances.  

The use of \textsc{Bayesian} methods to determine polarization observables can be expanded to any measurement of polarization observables and should provide more insight regarding the underlying posterior distributions and at the same time even uncover possible ambiguities. To increase the statistical precision of the existing measurement of the beam asymmetry in $\eta'$ photoproduction the decay channel $\eta'\to\pi^0\pi^0\eta$ should be investigated. Also, the use of machine learning methods to classify signal and background contributions in the selected data may be pursued. In the mean time, several upgrades of the CBELSA/TAPS experiment have been installed that promise to improve its trigger capabilities and data taking rates. The measurement of polarization observables in future beamtimes will profit from these improvements and help to further understand the nucleon excitation spectra.