%------------------------------------------------------------------------------
\chapter{Illustration of used software tools}
\section{ExPLORA}
Figure \ref{fig:xml} shows an example of a \texttt{.xml} file that was used to call the plugin that was written to select the reaction $\gamma p\to p\eta'\to p\gamma\gamma$. First of all, several files have to be included in order to acquire certain \emph{containers} that inhibit the raw data of the final state particles. Then the plugin is embedded with the options 
\begin{itemize}
	\item \texttt{MC} (bool) -- determines whether Monte Carlo or measured data are analyzed 
	\item \texttt{PWA} (bool) -- determines whether used Monte Carlo simulation have PWA weights
	\item \texttt{FOURGAMMAS} (bool) -- determines whether the generated final state has four photons or two
	\item \texttt{REALGAMMAS} (bool) -- determines whether real photons are part of the decay or not (e.g. $n\pi^+$)
	\item \texttt{allroutes} (CBTConfigString) -- gives the container that contains the routes of charged particles
\end{itemize}
\begin{figure}[htbp]
	\centering
	\includegraphics[width=\linewidth]{../demonstration/new_gg.png}
	\caption{Example \texttt{.xml} file that was used to call the plugin \texttt{CBTetaprimeanalysis.cpp} (line 20) with several self defined options.}
	\label{fig:xml}
\end{figure}
\section{Stan}
The simplest regression that can be made with \emph{Stan} is of the form
\begin{equation}
	y=a\cdot x +b+\epsilon.
\end{equation}
Here $y$ is a measured quantity with \textsc{Gaussian} noise $\epsilon$, $x$ are predictors and $a$ and $b$ are slope and intercept of a linear regression. Assuming each datapoint $y_n$ is independent and exhibits an individual noise term $\epsilon_n\sim\mathcal{N}(0,\sigma_n)$, the likelihood $p(y|a,b)$ can be formulated as
\begin{align}
	y_n\sim\mathcal{N}(a\cdot x_n +b,\sigma_n)&&\Leftrightarrow& &p(y|a,b)=\prod_{i=1}^N\mathcal{N}(y_n|a\cdot x+b,\sigma_n),
\end{align} 
if there are $N$ datapoints in total. Specifying e.g. normal priors for the two regression coefficients $a$ and $b$ completes the inference
\begin{align}
	a\sim\mathcal{N}(0,1)&&b\sim\mathcal{N}(0,1).
\end{align}
Figure \ref{fig:stan} shows the implementation of the described model in \emph{Stan}. First of all, all data that is read in has to be defined. Conveniently this can be done using the \texttt{vector} class, that corresponds to a list of e.g. all $x$ values. Next, the parameters of the model are defined and lastly the likelihood and priors are specified.
\begin{figure}[htbp]
	\centering
	\includegraphics[width=\linewidth]{../demonstration/stanfile.png}
	\caption{Example \texttt{.stan} file that can be used to perform a simple linear fit.}
	\label{fig:stan}
\end{figure}
\chapter{Additional plots and calculations}
\label{sec:app}
%------------------------------------------------------------------------------
This chapter will give additional calculations and plots which would have interrupted the train of thought unnecessarily in the main part. 
\section{Statistical error for the asymmetry $A(\phi)$}
\allowdisplaybreaks
\label{sec:stat_err}
Let $\tilde{N}^{\parallel/\bot}_i$ be the normalized event yields at bin $\phi_i$. As mentioned in section \ref{sec:meth}, the asymmetry $A_i$ at bin $i$ is then given by
\begin{equation}
	A_i=\frac{\tilde{N}^\bot_i-\tilde{N}^\parallel_i}{p_\gamma^\parallel\tilde{N}^\bot_i+p_\gamma^\bot\tilde{N}^\parallel_i}=\Sigma\cos\left(2\left(\alpha^\parallel-\phi_i\right)\right),
	\label{eq:evyieldasym_app}
\end{equation}
where the event yields are normalized over all $M$ $\phi$-bins $$\tilde{N}^{\parallel/\bot}_i=\frac{N_i^{\parallel/\bot}}{\sum_{j=1}^{M}N_j^{\parallel/\bot}}.$$
 To estimate statistical errors according to \textsc{Gaussian} error propagation, the partial derivatives with respect to $\tilde{N}_i^{\parallel/\bot}$ have to be built:
 \begin{equation}
 	\left(\Delta A_i\right)^2=\left(\frac{\partial A_i}{\partial \tilde{N}^\parallel_i}\Delta \tilde{N}^\parallel_i\right)^2+\left(\frac{\partial A_i}{\partial \tilde{N}^\bot_i}\Delta \tilde{N}^\bot_i\right)^2,
 \end{equation}
where
\begin{align}
	\left(\frac{\partial A_i}{\partial \tilde{N}^{\parallel/\bot}_i}\right)^2&=\left[\frac{\tilde{N}^{\bot/\parallel}_i\left(p_\gamma^\bot+p_\gamma^\parallel\right)}{\left(p_\gamma^\parallel\tilde{N}^{\bot}_i+p_\gamma^\bot\tilde{N}^\parallel_i\right)^2}\right]^2,\\
	&\text{ and with } \tilde{N}_i^{\bot/\parallel}=\tilde{N}_i\\
	\left(\Delta\tilde{N}_i\right)^2&=\left[\frac{\partial}{\partial N_i}\left(\frac{N_i}{\sum_{j}N_j}\right)\cdot\Delta N_i\right]^2+\sum_{j\neq i}\left[\frac{\partial}{\partial N_j}\left(\frac{N_i}{\sum_{j}N_j}\right)\cdot\Delta N_j\right]^2\\
	&=\left[\frac{\sum_{j\neq i} N_j}{\left(\sum_{j} N_j\right)^2}\cdot\Delta N_i\right]^2+\sum_{j\neq i}\left[-1\cdot\frac{N_i}{\left(\sum_{j}N_j\right)^2}\cdot\Delta N_j\right]^2\\
	&=\frac{1}{\left(\sum_{j} N_j\right)^4}\cdot\left[\left(\sum_{j\neq i} N_j \cdot\Delta N_i\right)^2+\sum_{j\neq i}\left(N_i\cdot\Delta N_j\right)^2\right].
\end{align}
One can then further use that $\left(\Delta N_i\right)^2  \approx N_i$. This holds only approximately, since the histograms are filled $N$ times with weights $w_n$ (see chapter \ref{sec:time}), but since the weights are either $w=1$ or $w\ll1$
\begin{equation}
	\Delta N_i=\sqrt{\sum_{n=1}^N w^2}\approx\sqrt{N_i}.
\end{equation}
\newpage
\section{Kinematic variables for each bin}
\label{sec:kinv}
\subsection{Coplanarity}
\begin{figure}[H]
	\centering
	\begin{subfigure}{\linewidth}
		\includegraphics[width=\linewidth]{../figs/hydrogen/bin_cuts/phicut_ebin1.pdf}
		\subcaption{$\SI{1500}{\mega\eV}\leq E_\gamma<\SI{1600}{\mega\eV}$}
	\end{subfigure}

	\begin{subfigure}{\linewidth}
		\includegraphics[width=\linewidth]{../figs/hydrogen/bin_cuts/phicut_ebin2.pdf}
		\subcaption{$\SI{1600}{\mega\eV}\leq E_\gamma<\SI{1700}{\mega\eV}$}
	\end{subfigure}
\caption{Coplanarity $\Delta\phi$ for all energy and angular bins. Data points are displayed as open circles, scaled Monte Carlo data belonging to $\eta'$ photoproduction is displayed as solid histogram. The determined cut ranges are indicated by the dashed red lines.}
\end{figure}
\begin{figure}[H]
\ContinuedFloat
	\begin{subfigure}{\linewidth}
		\includegraphics[width=\linewidth]{../figs/hydrogen/bin_cuts/phicut_ebin3.pdf}
		\subcaption{$\SI{1700}{\mega\eV}\leq E_\gamma<\SI{1800}{\mega\eV}$}
	\end{subfigure}
\caption{Coplanarity $\Delta\phi$ for all energy and angular bins. Data points are displayed as open circles, scaled Monte Carlo data belonging to $\eta'$ photoproduction is displayed as solid histogram. The determined cut ranges are indicated by the dashed red lines.}
\label{fig:appcopl}	
\end{figure}
Figure \ref{fig:appcopl} shows the coplanarity for all energy and angular bins. Cut ranges were determined from a \textsc{Gaussian} fit to the data points. Only slight dependency on beam energy and meson polar angle can be identified. Only $\eta'$ Monte Carlo data are fitted because the measured points do not give enough reference points for the fit to identify different contributing final states. There is good agreement between Monte Carlo simulations and measured data.
\newpage
\subsection{Polar angle difference}
\begin{figure}[H]
	\centering
	\begin{subfigure}{\linewidth}
		\includegraphics[width=\linewidth]{../figs/hydrogen/bin_cuts/thetacut_ebin1.pdf}
		\subcaption{$\SI{1500}{\mega\eV}\leq E_\gamma<\SI{1600}{\mega\eV}$}
	\end{subfigure}
	
	\begin{subfigure}{\linewidth}
		\includegraphics[width=\linewidth]{../figs/hydrogen/bin_cuts/thetacut_ebin2.pdf}
		\subcaption{$\SI{1600}{\mega\eV}\leq E_\gamma<\SI{1700}{\mega\eV}$}
	\end{subfigure}
\caption{Polar angle difference $\Delta\theta$ for all energy and angular bins. Data points are displayed as open circles, scaled Monte Carlo data belonging to $\eta'$ photoproduction is displayed as solid histogram. The determined cut ranges are indicated by the dashed red lines.}
\end{figure}
\begin{figure}[H]
	\ContinuedFloat
	\begin{subfigure}{\linewidth}
		\includegraphics[width=\linewidth]{../figs/hydrogen/bin_cuts/thetacut_ebin3.pdf}
		\subcaption{$\SI{1700}{\mega\eV}\leq E_\gamma<\SI{1800}{\mega\eV}$}
	\end{subfigure}
\caption{Polar angle difference $\Delta\theta$ for all energy and angular bins. Data points are displayed as open circles, scaled Monte Carlo data belonging to $\eta'$ photoproduction is displayed as solid histogram. The determined cut ranges are indicated by the dashed red lines.}
\label{fig:apptheta}	
\end{figure}
Figure \ref{fig:apptheta} shows the polar angle difference for all energy and angular bins. Cut ranges were determined from a \textsc{Gaussian} fit to the data points. Only slight dependency on beam energy can be identified whereas clear correlation between width of the distribution and meson polar angle exists. This is due to the hit detectors which exhibit different angular resolutions, as has been discussed in the main part. Only $\eta'$ Monte Carlo data are fitted because the measured points do not give enough reference points for the fit to identify different contributing final states. There is good agreement between Monte Carlo simulations and measured data.
\newpage
\subsection{Missing mass}
\begin{figure}[H]
	\centering
	\begin{subfigure}{\linewidth}
		\includegraphics[width=\linewidth]{../figs/hydrogen/bin_cuts/mismcut_ebin1.pdf}
		\subcaption{$\SI{1500}{\mega\eV}\leq E_\gamma<\SI{1600}{\mega\eV}$}
	\end{subfigure}
	
	\begin{subfigure}{\linewidth}
		\includegraphics[width=\linewidth]{../figs/hydrogen/bin_cuts/mismcut_ebin2.pdf}
		\subcaption{$\SI{1600}{\mega\eV}\leq E_\gamma<\SI{1700}{\mega\eV}$}
	\end{subfigure}
\caption{Missing mass $m_x$ for all energy and angular bins. Data points are displayed as open circles, scaled Monte Carlo data belonging to $\eta'$ (black), $2\pi^0$ (yellow) and $\pi^0\eta$ (magenta) photoproduction is displayed as solid histogram while their sum is displayed as turquoise histogram. The determined cut ranges are indicated by the dashed red lines.}
\end{figure}
\begin{figure}[H]
	\ContinuedFloat
	\begin{subfigure}{\linewidth}
		\includegraphics[width=\linewidth]{../figs/hydrogen/bin_cuts/mismcut_ebin3.pdf}
		\subcaption{$\SI{1700}{\mega\eV}\leq E_\gamma<\SI{1800}{\mega\eV}$}
	\end{subfigure}
	\caption{Missing mass $m_X$ for all energy and angular bins. Data points are displayed as open circles, scaled Monte Carlo data belonging to  $\eta'$ (black), $2\pi^0$ (yellow) and $\pi^0\eta$ (magenta) photoproduction is displayed as solid histogram while their sum is displayed as turquoise histogram. The determined cut ranges are indicated by the dashed red lines.}
	\label{fig:appmism}
\end{figure}
Figure \ref{fig:appmism} shows the missing mass for all energy and angular bins. Cut ranges were determined from a \textsc{Novosibirsk} fit to the Monte Carlo data. Only slight dependency on meson polar angle can be identified. Especially at higher beam energies the missing mass peak grows wider with flat background contributions from $2\pi^0$ and $\pi^0\eta$ production towards higher masses. The Monte Carlo fit mostly shows consistency with the fit of the invariant mass spectra. However, spectra are to be seen with caution, since the shapes of the two different background contributions are very similar and there is no other reference point in the missing mass spectrum as opposed to the invariant mass. Fits to the invariant mass spectra may reveal background contributions where a fit to the missing mass spectrum failed to find any. There is good agreement between Monte Carlo simulations and measured data.
\newpage
\subsection{Invariant mass}
\begin{figure}[H]
	\centering
	\begin{subfigure}{\linewidth}
		\includegraphics[width=\linewidth]{../figs/hydrogen/bin_cuts/invcut_ebin1.pdf}
		\subcaption{$\SI{1500}{\mega\eV}\leq E_\gamma<\SI{1600}{\mega\eV}$}
	\end{subfigure}
	
	\begin{subfigure}{\linewidth}
		\includegraphics[width=\linewidth]{../figs/hydrogen/bin_cuts/invcut_ebin2.pdf}
		\subcaption{$\SI{1600}{\mega\eV}\leq E_\gamma<\SI{1700}{\mega\eV}$}
	\end{subfigure}
\caption{Invariant mass $m_\text{meson}$ for all energy and angular bins. Data points are displayed as open circles, scaled Monte Carlo data belonging to $\eta'$ (black), $2\pi^0$ (yellow), $\pi^0\eta$ (magenta), $\pi^0$ (green) and $\omega$ (blue) photoproduction is displayed as solid histogram while their sum is displayed as turquoise histogram. The determined cut ranges are indicated by the dashed red lines.}
\end{figure}
\begin{figure}[H]
	\ContinuedFloat
	\begin{subfigure}{\linewidth}
		\includegraphics[width=\linewidth]{../figs/hydrogen/bin_cuts/invcut_ebin3.pdf}
		\subcaption{$\SI{1700}{\mega\eV}\leq E_\gamma<\SI{1800}{\mega\eV}$}
	\end{subfigure}
\caption{Invariant mass $m_\text{meson}$ for all energy and angular bins. Data points are displayed as open circles, scaled Monte Carlo data belonging to $\eta'$ (black), $2\pi^0$ (yellow), $\pi^0\eta$ (magenta), $\pi^0$ (green) and $\omega$ (blue) photoproduction is displayed as solid histogram while their sum is displayed as turquoise histogram. The determined cut ranges are indicated by the dashed red lines.}
\label{fig:appinv}
\end{figure}
Figure \ref{fig:appinv} shows the invariant mass for all kinematic bins. Hardly any dependence on meson direction and beam energy is observed. However, background contributions are especially observed in very forward and backward direction towards higher beam energies in consistency with findings from the missing mass spectra.  A flat background is realized by $2\pi^0$ and $\pi^0\eta$ production. There is good agreement between Monte Carlo simulations and measured data.
\chapter{Discussion of binned fits}
\label{app:binnedfits}
Investigation of toy Monte Carlo experiments (cf. section \ref{sec:sigma_etap}) revealed that the choice of binning leads to systematic errors regarding the parameter $\Sigma$ when fitting a binned distribution to the equation 
\begin{equation}
	A\left(\phi\right)=\Sigma\cdot\cos\left(2\left(\alpha^\parallel-\phi\right)\right).
\end{equation}
To investigate this further, the distributions $A\left(\phi\right)$ from different toy Monte Carlo experiments where fitted for several binnings in $\phi$. Three different Monte Carlo experiments were considered, each corresponding roughly to the expected statistics in one kinematic bin for $\pi^0$, $\eta$ and $\eta'$ photoproduction, respectively. For simplicity's sake, only least squares fits are shown here, although similar results were found for \textsc{Bayesian} fits also. The equivalency of \textsc{Bayesian} and least squares fit has been demonstrated sufficiently up until now. To identify the bias that is introduced by binning the data, 10000 toy Monte Carlo bins for each setting are fitted for $n=10,15,20,\dots,100$ bins. Then the dependence from the amount of bins of the mean $\mu$ of the normalized residuals $\xi$ as well as the mean $\chi^2$ value of all fits is investigated. This is shown in Figure \ref{fig:appbin}; The fitted mean of the normalized posteriors $\xi$ is plotted against the number of bins (blue data points, left ordinate) as well as the mean $\chi^2$ of 10000 fits depending on the number of bins (red data points, right ordinate). Clear dependencies can be made out: while too few bins tend to underestimate the true value of the beam asymmetry, too much bins will lead to an overcompensation. For this reason, the functions $\chi^2(n)$ and $\mu(n)$ are monotonously rising with increasing number of bins $n$. An exception is realized by the samples that simulated the statistics of the $\eta'$ final state, which can be explained by the fact, that after reaching a certain number of bins, no sensible fit estimates can be made anymore because too few data points are available. A minimum deviation from the nominal value is reached with $n=10,20,30$ bins for statistics comparable to $\eta',\eta$ and $\pi^0$ production respectively. This does not coincide with a minimum $\chi^2$ necessarily, although the mean $\chi^2$ values associated with the best estimation of the input value are compatible with 1. Figure \ref{fig:appbin} however remarkably shows the influence binning has on the extraction of the beam asymmetry. Since there exist other methods, binned fits should only be used as a sanity check, but generally avoided, to circumvent the introduction of any systematics inherent to binning.
\begin{figure}[htbp]
	\centering
	\includegraphics[width=\linewidth]{../bayes/toyMC/plots/binnedfits.pdf}
	\caption{Fit performance in dependence of the number of bins. Left axis shows the mean $\mu$ of the distribution of the normalized residuals $\xi$, right axis shows the mean $\chi^2$ of all fits. Squares simulate fits with statistics similar to the $\gamma p \to p\eta'\to p\gamma\gamma$ final state, triangles statistics similar to the $\gamma p\to p\eta\to p\gamma\gamma$ and final state, pentagons statistics similar to the $\gamma p \to p \pi^0\to p\gamma\gamma$. Dotted red line indicates the ideal value of $\chi^2=1$, while the dashed blue line indicates the ideal mean of the normalized residuals at $\mu=0$.}
	\label{fig:appbin}
\end{figure}
\chapter{Investigation of posteriors without truncation}
\label{app:trunc}
In section \ref{sec:sigma_etap} the investigation of posterior distributions from unbinned \textsc{Bayesian} fits was incomplete, since the normalized residuals as well as the likelihood pool could not be built with truncated posteriors. This is now supplemented here. After the fits have been repeated without implementing a truncation for the posteriors, all introduced measures to argue good fit quality can be examined. As a reminder, the data were generated with $\Sigma_1=0.5$ and $\Sigma_t=-0.5$. Figure \ref{fig:completpost} shows the combined posteriors of all fits. The left hand side shows the normalized residuals $\Xi$ and the right hand side the unnormalized combination of all posteriors. The results completely meet the expectations, the input values for the beam asymmetries are very well reproduced, and the normalized residuals follow a standard normal distribution as \textsc{Gaussian} fits show. Together with the results obtained from the independent likelihood pool (Figure \ref{fig:applik}), which is able to reproduce the input values within $1\sigma$, this suffices to conclude correct estimation of distribution widths with no inherent bias, as had already been found in section \ref{sec:sigma_eta}.

\begin{figure}[htbp]
	\begin{subfigure}{\linewidth}
		\includegraphics[width=.49\linewidth]{../bayes/etap_event_based_fit/plots/combined_post_add_notrunc.pdf}
		\includegraphics[width=.49\linewidth]{../bayes/etap_event_based_fit/plots/combined_post_add_notrunc_raw.pdf}
		\subcaption{Beam asymmetry $\Sigma_1$}
	\end{subfigure}
	\begin{subfigure}{\linewidth}
	\includegraphics[width=.49\linewidth]{../bayes/etap_event_based_fit/plots/combined_post_add_notrunc_bkg.pdf}
	\includegraphics[width=.49\linewidth]{../bayes/etap_event_based_fit/plots/combined_post_add_notrunc_bkg_raw.pdf}
	\subcaption{Background beam asymmetry $\Sigma_t$}
\end{subfigure}
	\caption{Combined posteriors of all 1000 fits without truncation for the signal beam asymmetry $\Sigma_1$ and the background beam asymmetry $\Sigma_t$. Left: normalized residuals $\Xi$, Right: unaltered added posterior distributions. \textsc{Gaussian} fits have been performed with results given on top of each plot.}
	\label{fig:completpost}
\end{figure}
\begin{figure}[htbp]
	\includegraphics[width=.49\linewidth]{../bayes/etap_event_based_fit/plots/combined_post_mul.pdf}
	\includegraphics[width=.49\linewidth]{../bayes/etap_event_based_fit/plots/combined_post_mul_bkg.pdf}
	\caption{Posterior distributions of $\Sigma_1$ (left) and $\Sigma_t$ (right) combined in an independent likelihood pool. \textsc{Gaussian} fits to the distribution confirm the reproduction of the input values within $1\sigma$. Note that only very few datapoints were available for the fits, because the distributions overwhelmingly converge into a single bin at $\pm0.5$, hence the large errors on the fit parameters.}
	\label{fig:applik}
\end{figure}





\paragraph{Remark:}
It turned out that without the amount of statistics that was used in section \ref{sec:sigmaeta}, normal priors centered at $0$ for the asymmetries $\Sigma_t$ and $\Sigma$ will mislead the fit results for these parameters towards $0$ if no lower and upper boundaries are used. Instead the priors were then chosen to be uniform on the interval $[-2,2]$. This imposes boundaries, but will not truncate the posteriors because the distributions are not expected to be this wide. All distributions shown here are generated using this model to complete the full investigation of posteriors. Previous toy Monte Carlo experiments (Figure \ref{fig:toyMCpost}) as well as very good agreement between point estimates and posterior distributions (Figure \ref{fig:sigmaetap}) together with the results shown here confirm the validity of the fit used in the main part.