% !TEX root = master_thesis.tex

%==============================================================================
\chapter{Introduction}
\label{sec:intro}
This chapter will give a brief introduction into the theoretical concepts that motivate this work. First the Standard Model of Particle Physics is sketched along with characteristics of the strong interaction. Next, the photoproduction of mesons and the importance of polarization observables is explained. Lastly the particular motivation and structure of this work will be presented.
%==============================================================================
\section{The Standard Model of Particle Physics}
The \emph{Standard Model of Particle Physics} (SM) is believed to describe the fundamental particles and forces of the universe. It distinguishes between \emph{fermions} and \emph{bosons}. While matter consists of fermions, bosons are particles that mediate the fundamental interactions. 

Fermions themselves are again grouped into (anti-)quarks and (anti-)leptons. There are three generations of quarks and leptons. Of these only the first and lightest consists of stable particles, i.e. the up and down quark as well as the electron and its neutrino. All other particles are heavier and not stable, they will thus decay fast via the strong, electromagnetic or weak interaction. A summary of properties of quarks and leptons is shown in table \ref{tab:sm0}.
\begin{table}[htbp]
	\centering
	\begin{tabular}{cccccc}
		\toprule
		&1st & 2nd & 3rd & el. charge& color charge\\
		\hline
		Quarks & $u$&$c$&$t$& 2/3 & r,g,b\\
		&$d$&$s$&$b$& 1/3 & r,g,b\\
		Leptons& $e$&$\mu$&$\tau$&-1& -\\
		& $\nu_e$&$\nu_\mu$&$\nu_\tau$&0&-\\
		\bottomrule
		
	\end{tabular}
\caption{Summary of the particles of the SM}
\label{tab:sm0}
\end{table}

In total, four interactions are part of the SM: strong, electromagnetic, weak and gravitational interaction \footnote{they are ordered here according to their relative strength}. In particle physics gravitation can be neglected. Strong, electromagnetic and weak interaction are transmitted by gluons $g$, photons $\gamma$ and $W/Z$-bosons respectively. Strong and weak interaction are restricted to a finite range of the order of the nucleon radius, whereas electromagnetic interaction and gravitation have infinite range. A summary of the SM can be found in table \ref{tab:sm1}.

 

\begin{table}[htbp]
	\centering
\begin{tabular}{ccc}
	\toprule
	interaction & couples to & Gauge boson\\
	\hline
	strong & color & $g$\\
	electromagnetic& electric charge & $\gamma$\\
	weak&weak charge & $W^\pm,Z^0$\\
	\bottomrule
\end{tabular}

\caption{Summary of the interactions of the SM}
\label{tab:sm1}
\end{table}

\section{Photoproduction of Pseudoscalar Mesons}
$$\int_0^\infty\frac{\sin \alpha\beta x}{\gamma x}$$
\section{Polarization Obervables and the Complete Experiment}
bla
\section{Motivation and Structure of this Thesis}
bla